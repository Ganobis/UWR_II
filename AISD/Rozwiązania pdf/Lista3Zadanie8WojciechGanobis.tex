\documentclass{article}
\usepackage[T1]{fontenc}
\usepackage{graphicx}
%\usepackage{pgfplots}
%\usepackage[OT4,plmath]
\usepackage{epstopdf}
\usepackage{amsmath,amssymb,amsfonts,amsthm}
\usepackage[utf8]{inputenc}
%\usepackage[english]{babel}
\newcommand\tab[1][1cm]{\hspace*{#1}}
\usepackage{hyperref}
\usepackage{color}
\usepackage{algorithm2e}
\usepackage[margin=1.25in]{geometry}
\usepackage[figurename=Wykres]{caption}
\DeclareMathSizes{12}{30}{16}{12}
%\usepackage[scaled=1.5]{helvet}

\title{Lista 3, Zadanie 8}
\author{Wojciech Ganobis 310519}
\date{13/06/20}

\begin{document}
\maketitle

W zadaniu dostaje dwie uporządkowanie niemalejąco tablice $T_{1}$ oraz $T_{2}$. Mamy znaleźć medianę dla wielozbioru powstałego z połączenia tych dwóch tablic. Liczba elementów w naszych tablicach jest równa $n$.\\

Oznaczmy sobie:\\
-$x$ jako mediane zbioru $T_{1}$\\
-$y$ jako mediane zbioru $T_{2}$\\

Jeśli $x < y$ to:\\
\tab Usuwamy z tablicy $T_1$  wszystkie elementy które są "poniżej" pozycji $x$, a z tablicy $T_2$  usuwamy $ceil(\frac{n}2) - 1$ "ostatnich elementów"\\
W przeciwnym porzypadku tablice zmniejszamy odwrotnie.\\

Wykonujemy działanie dopóki nasze tablice nie będą miały rozmiaru $4$ lub mniejszego. Wtedy znajdujemy mediane dwóch tablic metodą:(k to liczba posostałych elementów w tablicy)\\
\tab \tab 1. Wybieramy najmniejszy element z 2 tablic i usuwamy go.\\
\tab \tab 2. $k--$\\
\tab \tab 3.  Jeśli $k == 0$ to zwróć ostani usuniety element.\\
\tab \tab 4. Wróć do 1.\\

Nasz algorym działa w czasie $log(n)$(poza końcówką, która wykona maksymalnie 4 powtórzeniai więc nie biorę jej pod uwagę), ponieważ za każdym razem zmniejszamy tablice o połowę.



\end{document}
