\documentclass{article}
\usepackage[T1]{fontenc}
\usepackage{graphicx}
%\usepackage{pgfplots}
%\usepackage[OT4,plmath]
\usepackage{epstopdf}
\usepackage{amsmath,amssymb,amsfonts,amsthm}
\usepackage[utf8]{inputenc}
%\usepackage[english]{babel}
\newcommand\tab[1][1cm]{\hspace*{#1}}
\usepackage{hyperref}
\usepackage{color}
\usepackage{algorithm2e}
\usepackage[margin=1.25in]{geometry}
\usepackage[figurename=Wykres]{caption}
\DeclareMathSizes{12}{30}{16}{12}
%\usepackage[scaled=1.5]{helvet}

\title{Lista 5, Zadanie 4}
\author{Wojciech Ganobis 310519}
\date{13/06/20}

\begin{document}
\maketitle

Mamy dwa ciągi w postaci:\\
- $X = x_1, x_2, x_3, ..., x_n$\\
- $Y = y_1, y_2, y_3, ..., y_n$\\
Z tych dwóch ciągów tworzymy jeden wspólny w postaci:\\
$$Z_1 = x_1, y_1, x_2, y_2, ..., x_n, y_n$$
Teraz tworzeymy kolejne ciągi przez zamienianie elementów $z_i$ oraz $z_{i+1}$ (gdzie $i \in \{1, 2, ..., 2n-1\}$). Takich zmian w ciągu o długości $2n$ można wykonać $2n-1$ razy. Czyli w sumie mamy $2n$ ciągów.\\

Teraz mając $2n$ ciągów będziemy pytać adwersarza.Wiemy, że aby otrzymać odpowiedź musi pozostać nam dokładnie jeden ciąg $Z$. Mając $2n$ możliwych zestawów, musimy udowodnić, że zapytanie usuwa conajwyżej jeden ciąg $Z$.\\

Jedyne sensowne zapytania to "jak $x_i$ jest w stosunku do $y_j$".\\
Eliminacja odbywa się tylko wtedy gdy:\\
\tab - $i = j$. 

\end{document}
