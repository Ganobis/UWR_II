\documentclass{article}
\usepackage[T1]{fontenc}
\usepackage{graphicx}
%\usepackage{pgfplots}
%\usepackage[OT4,plmath]
\usepackage{epstopdf}
\usepackage{amsmath,amssymb,amsfonts,amsthm}
\usepackage[utf8]{inputenc}
%\usepackage[english]{babel}
\newcommand\tab[1][1cm]{\hspace*{#1}}
\usepackage{hyperref}
\usepackage{color}
\usepackage{algorithm2e}
\usepackage[margin=1.25in]{geometry}
\usepackage[figurename=Wykres]{caption}
\DeclareMathSizes{12}{30}{16}{12}
%\usepackage[scaled=1.5]{helvet}

\title{Lista 0, Zadanie 7}
\author{Wojciech Ganobis 310519}
\date{20/05/20}

\begin{document}
\maketitle

Idea: Do każdego wieszchołka w naszym drzewie dopisujemy czas wejśća oraz wyjścia DFS. Teraz mamy z definicji DFS'a wiemy, że jeśli czas wejścia $u \leq$ czas wejścia $v$ i czas wyjścia $u \leq$ czas wyjścia $v$ to $u$ należy do drzewa $v$.\\

Algorytm:\\

DFS(graf G):\\
\tab	czas =0;\\
\tab	ustaw wszystkie weiszchołki czas wejścia na $null$\\
\tab	odwiedź(korzeń, czas)\\
\\
Odwiedź($u$, czas):\\
\tab	czas wejścia = czas\\
\tab	dla każdego wieszchołka $v$ na liście sąsiedźtwa:\\
\tab\tab		jeśli($v$ czas wejścia == $null$) to\\
\tab\tab\tab			czas += 1\\
\tab\tab\tab			czas = Odwiedź($v$, czas)\\
\tab\tab	czas wyjścia = czas\\
\tab	return czas\\
\\
CzyLeży($v, u$):\\
\tab	jeśli(czas wejścia$u \leq$ cas wejścia $v \wedge$ czas wyjścia $u \leq$ czas wyjścia $v$)\\
\tab\tab		return T\\
\tab	else\\
\tab\tab		return F\\
\\

Czas działania algorytmu to czas dziaania DFS + O(1). Wiemy, że DFS jest najszybszym algorytmem przeszukiwania dzew(lub należy do najszybszysz inny taki algorytm to BFS), więc nie można wykonać tego działania szybciej.


\end{document}
