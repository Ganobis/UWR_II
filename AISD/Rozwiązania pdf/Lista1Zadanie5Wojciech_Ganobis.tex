\documentclass{article}
\usepackage[T1]{fontenc}
\usepackage{graphicx}
%\usepackage{pgfplots}
%\usepackage[OT4,plmath]
\usepackage{epstopdf}
\usepackage{amsmath,amssymb,amsfonts,amsthm}
\usepackage[utf8]{inputenc}
%\usepackage[english]{babel}
\newcommand\tab[1][1cm]{\hspace*{#1}}
\usepackage{hyperref}
\usepackage{color}
\usepackage{algorithm2e}
\usepackage[margin=1.25in]{geometry}
\usepackage[figurename=Wykres]{caption}
\DeclareMathSizes{12}{30}{16}{12}
%\usepackage[scaled=1.5]{helvet}

\title{Lista 1, Zadanie 5}
\author{Wojciech Ganobis 310519}
\date{20/05/20}

\begin{document}
\maketitle
Pierw sortuje wieszchołki topologicznie.\\\\
Potem używam algorytmu:\\
Droga(G):\\
\tab for $0\leq i \leq G.ilość_wieszchołków$\\
\tab\tab jeśli żadna krwędź nie wchodzi do wieszchoka to\\
\tab\tab\tab pozycja[i] = 0\\
\tab\tab\tab ojciec[i] = $null$\\
\tab\tab w przeciwnym przypadku\\
\tab\tab\tab sąsiad[]\\
\tab\tab\tab dla każdego wieszchołka m, takiego, że intnieje krawędź z m do następnego wieszchołka\\
\tab\tab\tab\tab dodaj do sąsiad[(m, pozycja[m])]\\
\tab\tab\tab max = max\_po\_drugim\_elemencie(sąsiada)\\
\tab\tab\tab ojciec[i] = max[0]\\
\tab\tab\tab pozycja[i] = max[1]+1\\
\tab ścieżka=[]\\
\tab koniec = 0\\
\tab maks  = 0\\
\tab for $0 \leq i \leq G.ilość_wieszchołków$\\
\tab\tab jeśli pozycja[i] > maks\\
\tab\tab\tab koniec = i\\
\tab\tab\tab maks = pozycja[i]\\\\
\end{document}
