\documentclass{article}
\usepackage[T1]{fontenc}
\usepackage{graphicx}
%\usepackage{pgfplots}
%\usepackage[OT4,plmath]
\usepackage{epstopdf}
\usepackage{amsmath,amssymb,amsfonts,amsthm}
\usepackage[utf8]{inputenc}
%\usepackage[english]{babel}
\newcommand\tab[1][1cm]{\hspace*{#1}}
\usepackage{hyperref}
\usepackage{color}
\usepackage{algorithm2e}
\usepackage[margin=1.25in]{geometry}
\usepackage[figurename=Wykres]{caption}
\DeclareMathSizes{12}{30}{16}{12}
%\usepackage[scaled=1.5]{helvet}

\title{Lista 6, Zadanie 4}
\author{Wojciech Ganobis 310519}
\date{13/06/20}

\begin{document}
\maketitle
 Nasza tablica będzie miała tylko elementy 1, 2 lub 3. Można więc prosto przesortować ją idąc pokolei po elementach wrzucając jedynki na przód lub wyrzucając trójki na koniec. Nasz algorytm będzie wyglądał następująco:\\
Dane:
\begin{itemize}
\item $T$ - tablica
\item $n$ - liczba elementów tablicy
\end{itemize}
$Alg(T, n)$:\\
\tab $n--$\\
\tab $a , b = 0$ (a oznacza elementy po których występuje tylko 2 lub 3, nastomiast b oznacza te elementy po których występuje tylko 3)\\
\tab dopóki $b < n$\\
\tab \tab jeśli $T[b] == 1$:\\
\tab \tab \tab $swap(T[a], T[b])$\\
\tab \tab \tab $a++, b++$\\
\tab \tab w przeciwynm przypadku jeśli $T[b] == 2$\\
\tab \tab \tab $b++$\\
\tab \tab w przeciwnym przypadku\\
\tab \tab \tab $swap(T[b], T[n])$\\
\tab \tab \tab $n--$\\

Po takim przejściu tablica jest posortowana. Złożoność algorytmu to $O(n)$, ponieważ przez każdy element tablicy przechodzimy raz.
\end{document}
