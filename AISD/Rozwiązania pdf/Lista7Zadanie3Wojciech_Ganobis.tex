\documentclass{article}
\usepackage[T1]{fontenc}
\usepackage{graphicx}
%\usepackage{pgfplots}
%\usepackage[OT4,plmath]
\usepackage{epstopdf}
\usepackage{amsmath,amssymb,amsfonts,amsthm}
\usepackage[utf8]{inputenc}
%\usepackage[english]{babel}
\newcommand\tab[1][1cm]{\hspace*{#1}}
\usepackage{hyperref}
\usepackage{color}
\usepackage{algorithm2e}
\usepackage[margin=1.25in]{geometry}
\usepackage[figurename=Wykres]{caption}
\DeclareMathSizes{12}{30}{16}{12}
%\usepackage[scaled=1.5]{helvet}

\title{Lista 7, Zadanie 3}
\author{Wojciech Ganobis 310519}
\date{13/06/20}

\begin{document}
\maketitle
W tym zadaniu mamy $n$ kluczy oraz $n$ tablic. Określmy sobie $X_i$ jako zmienną losową określającą czy tablica numer $i$ jest pusta. Prawdopodobieństwo takiego zdażenia dla pojedynczego klucza to $$\frac{n-1}{n}.$$ Jeśli mamy $n$ kluczy to prawdopodobieństwo będzie wynosić $$(\frac{n-1}{n})^{n}.$$ Teraz obliczmy to dla $n$ list, czyli wystarczy zrobić sumę prawdopodobieństw
$$\sum_{i=1}^{n}(\frac{n-1}{n})^{n} = n \cdot (\frac{n-1}{n})^{n} = n \cdot (1-\frac{1}{n})^{n} = ne^{-1} = \frac{n}{e}$$

\end{document}
