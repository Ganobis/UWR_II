\documentclass{article}
\usepackage[T1]{fontenc}
\usepackage{graphicx}
%\usepackage{pgfplots}
%\usepackage[OT4,plmath]
\usepackage{epstopdf}
\usepackage{amsmath,amssymb,amsfonts,amsthm}
\usepackage[utf8]{inputenc}
%\usepackage[english]{babel}
\newcommand\tab[1][1cm]{\hspace*{#1}}
\usepackage{hyperref}
\usepackage{color}
\usepackage{algorithm2e}
\usepackage[margin=1.25in]{geometry}
\usepackage[figurename=Wykres]{caption}
\DeclareMathSizes{12}{30}{16}{12}
%\usepackage[scaled=1.5]{helvet}

\title{Lista 4, Zadanie 2}
\author{Wojciech Ganobis 310519}
\date{13/06/20}

\begin{document}
\maketitle
Nasze rozwiązanie będzie podobne do tego z wykładu, jedyne co będzie inne to to, że możmy przyjśc do jakigoś pola z góry lub dołu.

Oznaczenia:\\
\begin{itemize}
\item wartość - wartość pola
\item koszt - minimalny koszt
\end{itemize}
Pierwszą kolumnę wypełniamy wartościamu pól.\\
Dla każdej kolejnej kolumny obliczmy koszty w następujący sposób:\\
\tab koszt(i,j) = min\{ koszt( i+1, j-1), koszt( i, j-1 ), koszt( i-1, j-1 ) \} + wartość[i,j]\\
Uwzględniamy także przejścia z góry i dołu:\\
\tab koszt( i, j ) = min\{ koszt( i, j ), ( koszt( i-1, j ) + wartosc[i,j] ), ( koszt( i+1, j ) + wartosc[i,j] ) \}\\

Teraz wyznaczamy ścieżkę. Bierzemy najmniejszą wartość z ostatniej komlumny, potem idziemy najmniejszymi wartościami z pól z których mogliśmy przyjść.

Złożoność $O(nm)$, ponieważ wykonujemy $m$ pętli, gdzie każda ma złożoność $n$.

\end{document}
