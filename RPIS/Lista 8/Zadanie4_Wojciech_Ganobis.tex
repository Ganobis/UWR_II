\documentclass[12pt]{article}
\usepackage{amsmath}
\usepackage{hyperref}
\usepackage[latin1]{inputenc}
\usepackage{graphicx}
\usepackage{enumerate}


\title{Zadanie 4}
\author{Wojciech Ganobis}
\date{05/05/20}

\begin{document}
\maketitle
Wiemy, ze zmienne $X_{1}, X_{2}, oraz X_{3}$ sa niezalezne i maja rozklad o dystrybuanie $F(x)$ i gestosci $f(x)$.
$$f(x) = (F(x))'$$
$$F_{(2)}(x) = P(X_{(2)} < x) = P(X_{1} < x) \cdot P(X_{2} < x) \cdot P(X_{3} > x) + $$$$+ P(X_{1} < x)\cdot P(X_{3} < x) \cdot P(X_{2} > x) +  P(X_{2} < x)\cdot P(X_{3} < x) \cdot P(X_{1} > x) +$$$$+  P(X_{1} < x)\cdot P(X_{2} < x) \cdot P(X_{3} < x) =$$$$ = F(x) \cdot F(x) \cdot (1-F(x)) +  F(x) \cdot F(x) \cdot (1-F(x)) + F(x) \cdot F(x) \cdot (1-F(x)) + F(x) \cdot F(x) \cdot F(x)=$$$$ = 3 \cdot F^{2}(x) \cdot (1-F(x)) + F^{3}(x)$$
\\
\\
$$f_{(2)} =( F_{(2)})' = 6F(x) \cdot (1-F(x)) \cdot f(x) + 3F^{2}(x) \cdot (-f(x)) + 3F^{2}(x) \cdot f(x) =$$$$= 6F(x) \cdot (1-F(x)) \cdot f(x) $$
Co mielismy udowodnic.



\end{document}
