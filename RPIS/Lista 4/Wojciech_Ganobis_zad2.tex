\documentclass[12pt]{article}
\usepackage{amsmath}
\usepackage{hyperref}
\usepackage{mathtools}

\title{Zadanie 2}
\author{Wojciech Ganobis}
\date{31/03/20}

\begin{document}
\maketitle
 Wiemy, ze aby byla to gestosc dwuwymiarowa, podwojna calka z ograniczen podanych w zadaniu $0 \leq x \leq 3, 1 \leq y \leq 2$, jest rowna 1 oraz funkcja na całym obszarze musi być dodatnia.

\begin{description}
\item[I)]
$\int_{x_{1}}^{x_{2}} \int_{y_{1}}^{y_{2}} f(x,y) dy dx$
$$\int_0^3 \int_1^2 Cxy+x+y dy dx = 1$$
$$\int_0^3 \int_1^2 Cxy+x+y dy dx = \int_0^3 \left( \int_1^2 Cxy dy + \int_1^2x dy + \int_1^2 y dy \right) dx =$$
$$ = \int_0^3 \left( Cx\frac{3}{2} + x + \frac{3}{2} \right) dx =$$
$$=\frac{3}{2}C \cdot \frac{9}{2} +\frac{9}{2} + \frac{9}{2} = \frac{27}{4}C + 9$$
$$\frac{27}{4}C + 9 = 1$$
$$C=\frac{-32}{27} $$

$f(x,y) = \frac{-32}{27}xy + x + y$

\item[II)]
$f(x,y) \leq 0$

Wezmy dla przykladu skrajne warotsci. Otrzymujemy:

$$f(3,2) = \frac{-32}{27} \cdot 3 \cdot 2 + 3 + 2 = \frac{-19}{9}$$
Niestety otrzymujemy wartosc ujemna, co oznacza, ze calka nie jest dodatnia na calym obszarze. Nie ma wiec takiej stalej C.
\end{description}
  

\end{document}
