\documentclass[12pt]{article}
\usepackage{amsmath}
\usepackage{hyperref}
\usepackage[latin1]{inputenc}
\usepackage{graphicx}
\usepackage{enumerate}


\title{Zadanie kolokwium 2}
\author{Wojciech Ganobis 310519, Bartosz Troszka 309912}
\date{05/05/20}

\begin{document}
\maketitle
Celem  zadania jest obliczenie wspolczynnika korelacji miedzy zachorowaniami, na podstawie dostarczonych danych. Najpierw omowmy metode zrobienia tego zadnaia.\\
Do wykonania zadnia uzyjemy wspolczynnika korelacji Pearsona. Wyraza sie on wzorem:
$$r_{xy} = \frac{\mathrm{cov}(r, y)}{\sigma_X\sigma_Y} = \frac{\sum_{i=1}^n (x_i - \overline{x})(y_i - \overline{y})}{\sqrt{\sum_{i=1}^n (x_i - \overline{x})^2} \sqrt{\sum_{i=1}^n (y_i - \overline{y})^2}} =$$$$= \frac{n\cdot \sum x_{i}y_{i} - \sum x_{i} \cdot \sum y_{i}}{\sqrt{[n \cdot \sum x^{2}_{i} - (\sum x_{i})^{2}]\cdot[n \cdot \sum y^{2}_{i} - (\sum y_{i})^{2}]}},$$
gdzie $r_{XY} \in [-1, 1]$. Im wieksza wartosc bezwzgledna, tym silniejsza zaleznosc miedzy cechami. Jesli $r_{XY} == 1$ to zaleznosc jest dokladnie liniowa. Natomiast jesli $r_{XY} == 0$ to nie ma liniowej zaleznosci pomiedzy cechami. Zaleznoscli mozna opisac za pomoca tabelki:\\\\
\begin{tabular}{|l|c|r|} \hline
Korelacje & Ujemne & Dodatnie  \\
\hline 
Slabe & -0.5 do 0.0 & 0.0 do 0.5 \\
\hline 
Silne & -1.0 do -0.5 & 0.5 do 1.0 \\
\hline 
\end{tabular}\\\\
Kolejna rzecza wazna w zadaniu jest wybranie punktu poczatkowego. "Pierwsze" dane w kazdym panstwie sa niedokladne oraz niezetelne. Zaczniemy wiec obliczanie korelacji gdy liczba zakazonych bedzie zwiekszac sie kazdego dnia.\\
Przejdzmy teraz do obliczenia pierwszej korelacji.\\ \\
\begin{itemize}
\item
Pierwsze panstwa miedzy ktorymi bedziemy szukac korelacji to Wlochy i Hiszpania. Jako pinkt poczatkowy dla Wloch wybierzemy 22 luty, poniewaz po tym dniu kazdego nastepnego dnia liczba zarazonych przybywa. Dla Hiszpani natomiast wybierzemy 27 luty, z tego samego powodu. Z racji ze dane mialy byc do 30 kwietnia punkt koncowy dla Hiszpani to 30 kwietnia, a dla Wloszech 25 kwiatnia, alby liczba dni byla taka sama. 
Do obliczenia wykorzystamy wzor $ \frac{n\cdot \sum x_{i}y_{i} - \sum x_{i} \cdot \sum y_{i}}{\sqrt{[n \cdot \sum x^{2}_{i} - (\sum x_{i})^{2}]\cdot[n \cdot \sum y^{2}_{i} - (\sum y_{i})^{2}]}}$.\\
Dla ulatwienia obliczen zrobimy tableke:\\
\begin{tabular}{|l|c|r|c|c|} \hline
x & y & $x_{i}y_{i}$ & $x_{i}^{2}$ & $y_{i}^{2}$  \\
\hline 
6  & 10 & 60 & 36 & 100\\
\hline 
67 & 13 & 871 & 4489 &169\\
\hline
48 & 7 & 336 & 2304 & 49\\
\hline
105 & 13 & 1365 & 11025 & 169\\
\hline
\ldots & \ldots & \ldots & \ldots & \ldots\\
\hline
\end{tabular}\\

\begin{tabular}{|l|c|r|c|c|} \hline
$\sum$ kolumna1 & $\sum$ kolumna2 & $\sum$ kolumna3 & $\sum$ kolumna4 & $\sum$ kolumna5  \\
\hline
192991 & 225045 & 988355044 & 810008099 & 1286773571\\
\hline
\end{tabular}\\
Teraz mozna latwo obliczyc wspolczunnik wstawiajac do wzoru:
$$\frac{64 \cdot 988355044 - 192991 \cdot 225045}{\sqrt{[64 \cdot 810008099 - (192991)^{2}]\cdot [64 \cdot 1286773571 - (225045)^{2}]}} =  0.8827529322$$
Wynik ten oznacza ze korelacja jest silna dodatnia.
\item
Obliczny teraz korealacje dla innych panstw. Wezmy np Polske oraz Francje. Za punkt poczatkowy Polski wezmy 7 marzec, a koncowy 3 maja. Dla francji bedzie to 27 luty i 24 kwietnia. Metoda ta sama wiec nie bede zapisywal obliczen tylko sam wynik ktory wynosi:
$$0,7512537726$$
Oznacza to rowniez ze korelacja jest silna, jednak slabsza niz miedzy Hiszpania a Wlochami.
\item
Obliczmy teraz korelacje miedzy naszymi sasiadami mianowicie miedzy Czechami a Slowacja. Za punkt poczatkowy Slowacji wezmy 5 marca, a koncowy 2 maja. W Czechach namiosast dzien pozniej czyli 6 marca do 3 maja. Korelacja wynosi:
$$0,60372796$$
Korelacja jest juz na skraju korelacji silnej i korelacji slabej.
\item
Porownjamy teraz USA oraz Polske, przypuszczam ze korelacja bedzie slaba, poniewaz po polsce wirus nie roprzestrzenial sie tak szybko z powodu wprowadzenia ograniczen oraz z powodu mniejszej ilosci zaludnienia. Sprawdzmy wiec matematycznie czy nasz przypuszczenia sa poprawne. W USA za pierwszy dzien wezmieemy 22 luty, a ostatni. W Polsce natomiast pierwszy bedzie 7 marca, a ostatni 3 maja. Nasz wynik to:
$$0,6353313598$$
Widzimy wiec ze mimo roznej gestosci zaludnienia korelacja zachodzi calkiem niezle. Jest nawet wieksza niz miedzy Czechami a Slowacja, co jest zaskakujace.
\item
Sprawdzmy teraz korelacje miedzy panstwami beienych panstw innych kontynentow. Wezmy za przyklad Brazylie oraz Azerbejdzan. W Azerbejdzanie wyniki bedziemy brac z dni od 13 marca do 3 maja, natomiast z Brazyli od 5 maja do 25 kwietnia. Nasz korelacja to:
$$0,1799759062$$
Jak widac korelacja jest fatalna, oznacza to ze dane nie sa zetelne, wynika to prawdopodobnie z miedzy innymi ilosci robionych testow.\\
\item
Sprawdzmy jeszcze korelacje miedzy USA a Azerbejdzanem. Azerbejdzan bedzie zawieral dni od 13 marca do 3 maja, a USA od 22 luty do 13 kwietnia. Wynik to:
$$-0,08316804048$$
Oznacza to ze korelacja praktycznie nie istnieje, jest bardzo bliska 0.
\end{itemize}
Z powyzszych obliczen mozna wywnioskowac ze zachodzi korelacja miedzy zachorowaniami miedzy wiekszoscia panstw. Jednak model nie jest uniwersalny co widac miedzy korelacja  Brazylii a Azerbejdzamu oraz Azerbejdzanu oraz USA. \\
Zadanie zostalo wykonane na wersje punktowa 10+4 poniewaz obliczono korelacje dla wiekszej ilosci panstw.


\end{document}
