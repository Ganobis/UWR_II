\documentclass[12pt]{article}
\usepackage{amsmath}
\usepackage{hyperref}
\usepackage{mathtools}
\usepackage{listings}
\usepackage{xcolor}

\lstset { %
    language=C++,
    backgroundcolor=\color{black!5}, % set backgroundcolor
    basicstyle=\footnotesize,% basic font setting
}

\title{Zadanie Kolokwium}
\author{Wojciech Ganobis, Bartosz Troszka}
\date{19/04/20}

\begin{document}
\maketitle
Dla ustalonego $t > 0$ obliczyć wartość całki 
$$ G(t) = \int_0^{t} exp(-\frac{x^{2}}2  ) dx $$
 używajac metody Romberga i złożonego wzoru trapezów.

Opiszmy metode Romberga.

Metoda Romberga jest metoda całkowania numerycznego.\\
Obliczajac wzorem otrzymujemy tablice Romberga która wyglada nastepujaco:\\ \\
$R_{0, 0}$\\
$R_{0, 1} R_{1, 0}$\\
$R_{0, 2} R_{1, 1} R_{2, 0}$\\
$...$
\\
Metode Romberga można opisać rekurencyjnie:
\[ \begin{cases} 
      R_{m,i} &: \frac{4^m\cdot R_{m-1,i+1}-R_{m-1,i}}{4^m-1} \\
      R_{0,i} &: R_{2^i}=h_i\cdot \sum_{k=0}^{2^i-1}\left(\frac{f(x_k)+f(x_{k+1})}{2}\right) \\
   \end{cases}
\]

Pierwsza kolumna obliczana jest metoda trapezów, druga to metoda simsona.
Naszym wynikiem bedzie ostatni element trójkata (prawy dolny róg $R_(n, 0)$), ponieważ ma najlepsze przybliżenie.

Trzeba teraz napisać program obliczajacy tablice Romberga, z wystarczajacym błedem przybliżenia.

Bład w naszym przybliżeniu:\\
$$bład   R_{n, 0} = O((\frac{1}{2^{n}} * (b - a)^{2})) =$$
$$= O((\frac{1}{2^{n}} * t)^{2})$$
$$\frac{t^{2}}{2^{2n}} < 10^{-4} $$
$$ 2^{2n} > 10^{4} * t^{2}$$
\\
Podstawiajac to powstałego wzoru wiemy już jak dobrać n w zależności od t.

Wzór Romberga w jezyku C++:\\
\begin{lstlisting}
h=xn-x0;
t[0][0]=h/2*((1/x0)+ (1/xn));
for(i=1;i<=p;i++){
    sl=pow(2,i-1);
    sm=0;
    for(j=1;j<=sl;j++){
        a=x0+(2*j-1)*h/pow(2,i);
        sm=sm+(1/a);
    }
    t[i][0]=t[i-1][0]/2+sm*h/pow(2,i);
}
for(i=1;i<=p;i++){
    for(j=1;j<=i && j<=q;j++){
        m=i-j;
        t[m+j][j]=(pow(4,j)*t[m+j][j-1]-t[m+j-1][j-1])/(pow(4,j)-1);
    }
}
printf("%f",t[p][q]);
\end{lstlisting}
Gdzie :\\
\begin{description}
\item[a)] x0 = poczatek przedziału
\item[b)] xn = koniec przedziału
\item[c)] t = tablica Romberga
\end{description}


\end{document}
