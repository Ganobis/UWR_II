\documentclass[12pt]{article}
\usepackage{amsmath}
\usepackage{hyperref}
\usepackage[latin1]{inputenc}
\usepackage{graphicx}
\usepackage{enumerate}


\title{Zadanie 10}
\author{Wojciech Ganobis 310519}
\date{19/05/20}

\begin{document}
\maketitle

Wiemy, ze prosta regresmi wyraza sie wzorem:
$$y = a\cdot x + b$$
gdzie,
$$b = \overline{y} - a\overline{x}; a = \frac{\sum^{n}_{i=1}x_{i}y_{i} - n\overline{x}\overline{y}}{\sum^{n}_{i=1}x^{2}_{i}-nx^{2}}$$
Wiec liczyny:

\begin{itemize}
\item $\overline{x} = \frac{1}{8}\cdot\sum^{8}_{i=1}x_{i}=\frac{56}{8}=7$
\item $\overline{y} = \frac{1}{8}\cdot\sum^{8}_{i=1}y_{i}=\frac{50}{8}=5$
\item $\sum^{8}_{i=1}x_{i}y_{i}=364$
\item $\sum^{8}_{i=1}x^{2}_{i}=524$
\item $a = \frac{\sum^{n}_{i=1}x_{i}y_{i} - n\overline{x}\overline{y}}{\sum^{n}_{i=1}x^{2}_{i}-nx^{2}} = \frac{364-8\cdot7\cdot5}{524-8\cdot7^{2}}=\frac{84}{132}=\frac{7}{11}$
\item $b = \overline{y} - a\overline{x} = 5-7\cdot\frac{7}{11} = \frac{6}{11}$
\end{itemize}

Nasz otrzymany wynik to:
$$y=\frac{7}{11}x+\frac{6}{11}$$

\end{document}
